\documentclass[conference]{IEEEtran}
\IEEEoverridecommandlockouts
% The preceding line is only needed to identify funding in the first footnote. If that is unneeded, please comment it out.
\usepackage{cite}
\usepackage{amsmath,amssymb,amsfonts}
\usepackage{mathtools}
\usepackage{algpseudocode}
\usepackage{algorithm}
\usepackage{algorithmicx}
\usepackage{microtype}
\usepackage{float}
\usepackage{adjustbox}
\usepackage{booktabs,makecell,tabularx}
\usepackage{graphicx}
\usepackage{textcomp}
\usepackage{verbatim}
\usepackage{xcolor}
\usepackage{graphics} % for pdf, bitmapped graphics files
\usepackage{epsfig} % for postscript graphics files
\usepackage{mathptmx} % assumes new font selection scheme installed
\usepackage{multicol}
\usepackage[english]{babel}
\usepackage[T1]{fontenc}
\restylefloat{table}
\def\BibTeX{{\rm B\kern-.05em{\sc i\kern-.025em b}\kern-.08em
    T\kern-.1667em\lower.7ex\hbox{E}\kern-.125emX}}
\begin{document}
\title{Swarm and Evolutionary Based Algorithms used for Optimization}
\author{
	\IEEEauthorblockN{Augusto Mathias Adams\IEEEauthorrefmark{1}, Caio Phillipe Mizerkowski\IEEEauthorrefmark{2}, Christian Piltz Araújo\IEEEauthorrefmark{3} and Vinicius Eduardo dos Reis\IEEEauthorrefmark{4}}\\
	\IEEEauthorblockA{\IEEEauthorrefmark{1}GRR20172143, augusto.adams@ufpr.br, \IEEEauthorrefmark{2}GRR20166403, caiomizerkowski@gmail.com,\\ \IEEEauthorrefmark{3}GRR20172197, christian0294@yahoo.com.br, \IEEEauthorrefmark{4}GRR20175957, eduardo.reis02@gmail.com}
}

\maketitle


\begin{abstract}
    In this paper, a study of evolution and swarm based algorithms is presented,
    using two classical engineering problems: \textit{Spring Tension} and \textit{Pressure Vessel
    Designs}. The test code for the problems was made using the \textit{Python Language},
    version 3.10 and uses \textit{MealPy} package, version 2.5.1,  to provide the algorithms.
    Thre algorithms were randomly chosen from a vaste list from MealPy algorithms:
    \textit{Evolutionary Programming (LevyEP)}, \textit{Evolution Strategies (OriginalES)} and
    \textit{Genetic Algorithm (BaseGA)} from \textit{evolutionary\_based} subpackage;
    \textit{Bees Algorithm (OriginalBeesA)}, \textit{Firefly Algorithm (OriginalFFA)} and
    \textit{Particle Swarm Optimization (OriginalPSO)} from \textit{swarm\_based} subpackage.
    Each problem was modeled using \textit{SciPy} package, with constraints implemented as
    \textit{penalty functions}. Each algorithm were optimized separately to extract the best
    solutions from each problem using the \textit{MealPy\s} \textit{Tuner} utility.
    The results, however, are dependant of algorithm and problem presented but from
    similarity test using Friedman's Chi-Squared test, it would be suffice to say that using any
    of the tested algorithm will produce similar results in these two optimization problems.
\end{abstract}

\begin{IEEEkeywords}
	Optimization Methods, Evolutionary Programming, Evolutionary and Swarm Based Strategies.
\end{IEEEkeywords}

\section{Introduction}
\label{sec:introduction}

The study comprizes two great areas of artificial intelligence: Evolutionary
and Swarm Inteligence algorithms.

Evolution: From Jason Brownlee's \textit{``Clever Algorithms''} -
\textit{Evolutionary Algorithms belong to the Evolutionary Computation
field of study concerned with computational methods inspired by the process
and mechanisms of biological evolution. The process of evolution by
means of natural selection (descent with modification) was proposed by
Darwin to account for the variety of life and its suitability (adaptive
fit) for its environment. The mechanisms of evolution describe how
evolution actually takes place through the modification and propagation
of genetic material (proteins). Evolutionary Algorithms are concerned
with investigating computational systems that resemble simplified ver-
sions of the processes and mechanisms of evolution toward achieving
the effects of these processes and mechanisms, namely the development
of adaptive systems. Additional subject areas that fall within the realm
of Evolutionary Computation are algorithms that seek to exploit the
properties from the related fields of Population Genetics, Population
Ecology, Coevolutionary Biology, and Developmental Biology.}

Swarm Optimization: From Jason Brownlee's \textit{``Clever Algorithms''} -
\textit{Swarm intelligence is the study of computational systems inspired by
the ‘collective intelligence’. Collective Intelligence emerges through the
cooperation of large numbers of homogeneous agents in the environment.
Examples include schools of fish, flocks of birds, and colonies of ants.
Such intelligence is decentralized, self-organizing and distributed through
out an environment. In nature such systems are commonly used to
solve problems such as effective foraging for food, prey evading, or
colony re-location. The information is typically stored throughout the
participating homogeneous agents, or is stored or communicated in
the environment itself such as through the use of pheromones in ants,
dancing in bees, and proximity in fish and birds.}
