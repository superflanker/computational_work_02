Developed by F. Wilcoxon in 1945, the paired Wilcoxon test is based on the ranks
of intrapair differences. This non-parametric test, used to compare related samples,
is an alternative to the Student t-test when samples do not follow a normal distribution.
Therefore, the Wilcoxon test is used to test whether sample medians are equal in cases
where the assumption of normality is not satisfied or when it is not possible to check
this assumption.

Interpretation is as follows:
\begin{itemize}
    \item Assume two hypothesis about the data:
        \subitem $H_0$: the distributions of both samples are equal;
        \subitem $H_1$: the distributions of both samples are not equal.
    \item Given p-value, decide:
        \subitem p-value $\leq 0.05$: Reject $H_1$ with $95\%$ of confidence and accept null hypothesis;
        \subitem Accept $H_1$ otherwise.
\end{itemize}

The Table~\ref{wilcoxon_test:pressure_vessel_problem_original} gives the Wilcoxon test results
for \textit{Pressure Vessel Design} using the original variables bounding as exposed in Section~\ref{subsubsec:methodology-pressure-vessel-design}.

\begin{table}[H]
\centering
\caption{Significance Test Using Wilcoxon Test for Pressure Vessel Design (Original)}
\label{wilcoxon_test:pressure_vessel_problem_original}
\resizebox{\columnwidth}{!}{%
\begin{tabular}{lllllll}
\toprule
  --- & EP & ES & GA & BeesA & FFA & PSO \\
\midrule
   EP & 0.0 & 1.20e-16 & 3.90e-18 & 3.90e-18 & 0.07 & 2.02e-4 \\
   ES & 1.20e-16 & 0.0 & 6.56-06 & 4.96e-18 & 5.49e-14 & 1.41e-17 \\
   GA & 3.90e-18 & 6.56e-06 & 0.0 & 3.90e-18 & 3.90e-18 & 3.90e-18 \\
BeesA & 3.90e-18 & 4.96e-18 & 3.90e-18 & 0.0 & 3.90e-18 & 3.90e-18 \\
  FFA & 0.07 & 5.49e-14 & 3.90e-18 & 3.90e-18 & 0.0 &  5.34e-05 \\
  PSO & 2.02e-4 & 1.41e-17 & 3.90e-18 &  3.90e-18 & 5.34e-05 & 0.0 \\
\bottomrule
\end{tabular}}
\end{table}

It is easy to notice that, from Wilcoxon test, FAA and EP have different distributions.
$H_1$ is only accepted in case of FFA and EP: they distributions seems to be not equal.