
The Table~\ref{wilcoxon_test:pressure_vessel_problem} gives the Wilcoxon test results
for \textit{Pressure Vessel Design} using the proud-and-lame-homemade variables bounding
as exposed in Section~\ref{subsubsec:methodology-pressure-vessel-design}.


\begin{table}[H]
\centering
\caption{Significance Test Using Wilcoxon Test for Pressure Vessel Design}
\label{wilcoxon_test:pressure_vessel_problem}
\resizebox{\columnwidth}{!}{%
\begin{tabular}{lllllll}
\toprule
  --- & EP & ES & GA & BeesA & FFA & PSO \\
\midrule
   EP & 0.0 & 9.04e-18 & 3.90e-18 & 6.12e-18 & 8.79e-12 & 4.40e-18 \\
   ES & 9.04e-18 & 0.0 & 1.71e-14 & 1.64e-13 & 4.81e-18 & 3.90e-18 \\
   GA & 3.90e-18 & 1.71e-14 & 0.0 & 0.38 & 3.90e-18 & 3.90e-18 \\
BeesA & 6.12e-18 & 1.64e-13 & 0.38 & 0.0 & 4.30e-18 & 3.90e-18 \\
  FFA & 8.79e-12 & 4.81e-18 & 3.90e-18 & 4.30e-18 & 0.0 & 4.02e-18 \\
  PSO & 4.40e-18 & 3.90e-18 & 3.90e-18 & 3.90e-18 & 4.02e-18 & 0.0 \\
\bottomrule
\end{tabular}}
\end{table}

EP, ES, FAA and PSO appears to have the same distributions.
However, BeesA and GA appears to have different distributions.
This fact confirms the results found in Table~\ref{function_values:pressure_vessel_problem}.

