The Friedman test is the non-parametric test used to compare related sample data, that is,
when the same individual is evaluated more than once.
The Friedman test does not use the data numbers directly, but the ranks occupied by
them after the sorting done for each group separately. After sorting, the hypothesis
of equality of the sum of the ranks of each group is tested.

Interpretation is as follows:
\begin{itemize}
    \item Assume two hypothesis about the data:
        \subitem $H_0$: The mean for each population is equal.
        \subitem $H_1$: At least one population mean is different from the rest.
    \item Given p-value, decide:
        \subitem p-value $\geq 0.05$: Reject $H_1$ with $95\%$ of confidence and accept null hypothesis;
        \subitem Accept $H_1$ otherwise.
\end{itemize}

The Table~\ref{friedman_test} resumes the Friedman's Chi-Squared Test made using
the 6 samples took by the algorithm's best results fore each run:

\begin{table}[H]
\centering
\caption{Significance Test Using Friedman Chi-Squared Test}
\label{friedman_test}
\resizebox{\columnwidth}{!}{%
\begin{tabular}{lrr}
\toprule
                          Problem &   Rank &    p-value \\
\midrule
Pressure Vessel Design (Original) & 425.07428571 & 0.00000000 \\
           Pressure Vessel Design & 436.16000000 & 0.00000000 \\
            Spring Tension Design & 339.05714286 & 0.00000000 \\
\bottomrule
\end{tabular}}
\end{table}

Using the interpretation rules, since the p-value is less than $0.05$, the null hypotesis $H_0$
is rejected and each problem have at least one population with different mean from the rest.
Thus, not all algorithms gives the same result for the same problem.

