

\section{Discussion}

In this paper were tested 3 evolutionary algorithms (EP, ES and GA) and 3
swarm based algorithms (BeesA, FFA and PSO), implemented by the \textit{MealPy}
package. The chosen test problem were \textit{Pressure Vessel Design} and
\textit{Spring Tension Design} because they are well studied engineering problems
and are founs in many, if not all, optimization books everywhere. Thus, there are
more results to compare.

Both problems were implemented in full form, that is, with constraints, using the scheme of
\textit{penalty functions}, as it is recomended by \textit{MealPy}'s documentation.

Surprisingly, the \textit{Pressure Vessel Design} problem was a tricky one to solve.
Many articles point out the boundings as described for original problem
in Section~\ref{subsubsec:methodology-pressure-vessel-design}.
But, for some unknown reason, the same variable bounding does not work for this
implementation, and it was necessary to tweak the search space to give at least
comparable results with the literature.
It is not intended to point out errors in modelling or blame anyone, it is only a mere
note to a misbehaviour in solving a specific problem.

The results gives a notion that there are no panacea or universal solution for problems that
nothing is known except the cost function and constraints.
One algorithm's success to solve a specific problem is not a guarantee to solve anything
and so on.

At first, the \textit{Spring Tension Design} is easier to solve and is well behaved in the
given variable bounsings. But, at it is noticed in the Table~\ref{best_fits:spring_problem},
the cost function $f_x$ seems to have a plateau of weak minimizers, that is, values that gives
near te same results. The statistical parameters of the cost function values at the best fits
, as well the Wilcoxon test results fot this problem shows, there are no difference between the
distribution of each algorithm solution, but the Friedman Test (and clearly the boxplot in
figure~\ref{fig:spring_tension_design_boxplot}) indicates that they have significant
differences in the mean value. So, there are some algorithms not fitted for the problem's solution.
The best fit at all can be obtained using FFA algorithm, according to this paper.

The \textit{Pressure Vessel Design} was solver using two bounding schemes to show discrepancies
between the solutions.
First, it was solved using the literature boundings: the behaviour of the
solutions seems bizarre and make it evident that or the literature is wrong or the implementation
used in this paper is wrong.
Again, it is not intended heve to blame anyone.
The second solution, using the proud-and-lame-homemade
version of variable boundings seems to behave well.
The same behaviour of \textit{Spring Tension Design} is noticed in the tests, both statistical
, Wincoxon and Friedman Tests.

From similarity tests, the tests gave the following interpretations:

\begin{itemize}
    \item The best strategies to solve \textit{Pressure Vessel Design} in its proud-and-lame-homemade version
    are EP, ES, FFA and PSO, due to its stability and results similarity. The best of all is FFA due to its
    lower standard deviation.
    \item No strategy tested in this paper were able to solve \textit{Pressurew Vessel Design} in its original
    version and the available tools chosen for implementation.
    \item The best strategies to solve \textit{Spring Tension Design} are EP, ES, BeesA, FFA and PSO. The best of
    all is FFA due to its lower standard deviation.
\end{itemize}

\end{document}